\documentclass{article}

\usepackage[a4paper,margin=2.5cm]{geometry}
\usepackage{tikz}
\usepackage{amsmath}
\usepackage{fancyhdr}
\usepackage{hyperref}
\usepackage{float}
\usepackage{subcaption}
\usepackage{forest}
\usepackage{xcolor}
\usepackage{multirow}
\usepackage[shortlabels]{enumitem}

\title{\vspace{-2cm}ECON20005 Assignment 1}
\date{\today}
\author{Lucas Fern (1080613)\\Friday 12:00pm Tutorial}

\lhead{ECON20005 - Assignment 1}
\rhead{Lucas Fern (1080613)}
\pagestyle{fancy}

\begin{document}
\maketitle
\paragraph{Notation Convention} In this assignment $(\mbox{ANZ}, \mbox{ANZ})$ represents a strategy where the player chooses ANZ in in two subgames. Sets of strategies appear in curly braces - \{\}.
\section*{Question 1: Marriage and banking}
\subsection*{1.1}
\begin{center}
    \begin{forest}
        for tree={circle,s sep = 35pt,calign=center}
        [Tommy,
         [Gina,edge={->, red, line width=1.5pt},edge label={node[above left,midway,font=\scriptsize]{ANZ}}
          [{(22, 25)},edge={->, red, line width=1.5pt},edge label={node[midway,above left,font=\scriptsize]{ANZ}}]
          [{(18, $M^2 = 16$)},edge label={node[midway,above right,font=\scriptsize]{NAB}}]
         ]
         [Gina,,edge label={node[midway,above right,font=\scriptsize]{NAB}}
          [{(18, 25)},edge={->, red, line width=1pt},edge label={node[midway,above left,font=\scriptsize]{ANZ}}]
          [{(22, $M^2 = 16$)},edge label={node[midway,above right,font=\scriptsize]{NAB}}]
         ]
        ]
    \end{forest}
\end{center}

\subsection*{1.2}
\begin{enumerate}[label=\textbf{\alph*}.]
    \item Tommy and Gina.
    \item \mbox{NAB} and \mbox{ANZ}.
    \item It is a sequential game.
    \item There is perfect information.
\end{enumerate}

\subsection*{1.3}
Tommy has the set of strategies $\{(\mbox{ANZ}), (\mbox{NAB})\}$.\\[2mm]
Gina has $\{(\mbox{ANZ}, \mbox{ANZ}), (\mbox{ANZ}, \mbox{NAB}), (\mbox{NAB}, \mbox{ANZ}),(\mbox{NAB}, \mbox{NAB})\}$.

\subsection*{1.4}
For $M=4$, Gina's strategy is $\{(\mbox{ANZ}, \mbox{ANZ})\}$ since $25 > M^2 = 16$. By pruning the tree we resolve that Tommy's strategy is $\{(\mbox{ANZ})\}$.\\[2mm]
Therefore the equilibrium strategies for $M=4$ are Tommy: $\{(\mbox{ANZ})\}$, and Gina: $\{(\mbox{ANZ}, \mbox{ANZ})\}$. The equilibrium path is $\mbox{ANZ} \rightarrow \mbox{ANZ}$, and the equilibrium strategies are illustrated with {\color{red}red} arrows in Question 1.1.

\subsection*{1.5} In this game table, Tommy's best action for each of Gina's options is \underline{\textbf{underlined and bold}}. To assist with Question 1.6 Gina's best actions for $M<5$ are highlighted {\color{red}red}, and {\color{blue}blue} for $M>5$.
\begin{table}[H]
    \centering
    \begin{tabular}{cccccc}
                                                    &                          & \multicolumn{4}{c}{Gina}                                                                                                              \\ \cline{3-6} 
                                                    & \multicolumn{1}{c|}{}    & \multicolumn{1}{c|}{(ANZ, ANZ)} & \multicolumn{1}{c|}{(ANZ, NAB)} & \multicolumn{1}{c|}{(NAB, ANZ)} & \multicolumn{1}{c|}{(NAB, NAB)} \\ \cline{2-6} 
        \multicolumn{1}{c|}{\multirow{2}{*}{Tommy}} & \multicolumn{1}{c|}{ANZ} & \multicolumn{1}{c|}{\underline{\textbf{22}}, {\color{red}25}}     & \multicolumn{1}{c|}{\underline{\textbf{22}}, {\color{red}25}}     & \multicolumn{1}{c|}{\underline{\textbf{18}}, {\color{blue}$M^2 = 16$}}  & \multicolumn{1}{c|}{18, {\color{blue}$M^2 = 16$}}  \\ \cline{2-6} 
        \multicolumn{1}{c|}{}                       & \multicolumn{1}{c|}{NAB} & \multicolumn{1}{c|}{18, {\color{red}25}}     & \multicolumn{1}{c|}{\underline{\textbf{22}}, {\color{blue}$M^2 = 16$}}  & \multicolumn{1}{c|}{\underline{\textbf{18}}, {\color{red}25}}     & \multicolumn{1}{c|}{\underline{\textbf{22}}, {\color{blue}$M^2 = 16$}}  \\ \cline{2-6} 
    \end{tabular}
\end{table}
\noindent For $M=4$ (equivalent to the general case of $M<5$), 3 Nash equilibria exist for the game, these are:
\begin{enumerate}
    \item Tommy: $\{(\mbox{ANZ})\}$, and Gina: $\{(\mbox{ANZ}, \mbox{ANZ})\}$,
    \item Tommy: $\{(\mbox{ANZ})\}$, and Gina: $\{(\mbox{ANZ}, \mbox{NAB})\}$; and,
    \item Tommy: $\{(\mbox{NAB})\}$, and Gina: $\{(\mbox{NAB}, \mbox{ANZ})\}$.
\end{enumerate}
Only number 1 is subgame perfect, since although all 3 are Nash equilibria, the second and third involve one of the players making sub-optimal choices on one of their decision nodes.

\subsection*{1.6}
At $M=5$ Gina is indifferent between the banks as she receives an equal payoff of 25 regardless of her choice. This results in the 6 SPNE strategies:
\begin{itemize}
    \item Tommy: $\{(\mbox{ANZ})\}$, and Gina: $\{(\mbox{ANZ}, \mbox{ANZ})\}$,
    \item Tommy: $\{(\mbox{ANZ})\}$, and Gina: $\{(\mbox{ANZ}, \mbox{NAB})\}$,
    \item Tommy: $\{(\mbox{ANZ})\}$, and Gina: $\{(\mbox{NAB}, \mbox{ANZ})\}$,
    \item Tommy: $\{(\mbox{NAB})\}$, and Gina: $\{(\mbox{ANZ}, \mbox{NAB})\}$,
    \item Tommy: $\{(\mbox{NAB})\}$, and Gina: $\{(\mbox{NAB}, \mbox{ANZ})\}$; and,
    \item Tommy: $\{(\mbox{NAB})\}$, and Gina: $\{(\mbox{NAB}, \mbox{NAB})\}$.
\end{itemize}
Even though Tommy receives a payoff of 18 in some games and 22 in others, each of these is a SPNE, as Tommy cannot make a decision that will guarantee him a payoff of 22, since this relies entirely on Gina's choice and she is indifferent between the two.

\section*{Question 2: Iteratively eliminating dominated strategies}
\subsection*{2.1} Doing iterative elimination of dominated strategies:
\begin{enumerate}
    \item Player B eliminates R and W since they are dominated by L.
    \item Player A eliminates Q, R and W since they are dominated by L.
    \item Player B eliminates G and Q since they are dominated by K.
    \item Player A eliminates G as it is dominated by K.
\end{enumerate}
This reflects in the original game table as:
\begin{center}
    \includegraphics[width=0.7\linewidth]{iterative-elimination.png}
\end{center}
\newpage
\noindent And results in the normal form game:
\begin{table}[h!]
    \centering
    \begin{tabular}{cccc}
                                                   &                        & \multicolumn{2}{c}{Player B}                          \\ \cline{3-4} 
                                                   & \multicolumn{1}{c|}{}  & \multicolumn{1}{c|}{K}    & \multicolumn{1}{c|}{L}    \\ \cline{2-4} 
    \multicolumn{1}{c|}{\multirow{2}{*}{Player A}} & \multicolumn{1}{c|}{K} & \multicolumn{1}{c|}{{\color{red}5}, 3} & \multicolumn{1}{c|}{4, {\color{blue}4}} \\ \cline{2-4} 
    \multicolumn{1}{c|}{}                          & \multicolumn{1}{c|}{L} & \multicolumn{1}{c|}{3, {\color{blue}6}} & \multicolumn{1}{c|}{{\color{red}7}, 3} \\ \cline{2-4} 
    \end{tabular}
\end{table}

\subsection*{2.2}
In the table that emerges from Question 2.1 each players' best responses have been highlighted. From this we can see that there are \textit{no} pure-strategy Nash equilibria.

\subsection*{2.3}
The cell corresponding to both players choosing K has a sum of payoffs $5+3=8$, the cell with both players choosing L has sum of payoffs $7+3=10$. $8 \neq 10$, so this is not a zero sum game.

\section*{Question 3: Paul and Ringo take a vacation}
\subsection*{3.1}
\textbf{Extensive form:}
\begin{figure}[H]
    \centering
    \begin{forest}
        for tree={circle,s sep = 35pt,calign=center}
        [Paul,
         [Ringo,edge={->, red, line width=1.5pt},name=rleft,edge label={node[above left,midway,font=\scriptsize]{Gold Coast}}
          [{(5, 3)},edge={->, red, line width=1.5pt},edge label={node[midway,above left,font=\scriptsize]{Gold Coast}}]
          [{(-2, -2)},edge={->, blue, line width=1pt},edge label={node[midway,above right,font=\scriptsize]{Queenstown}}]
         ]
         [Ringo,edge={->, blue, line width=1.5pt},name=rright, edge label={node[midway,above right,font=\scriptsize]{Queenstown}}
          [{(-2, -2)},edge={->, red, line width=1pt},edge label={node[midway,above left,font=\scriptsize]{Gold Coast}}]
          [{(1, 6)},edge={->, blue, line width=1.5pt},edge label={node[midway,above right,font=\scriptsize]{Queenstown}}]
         ]
        ]
        \draw[dashed] (rleft) to[out=0,in=180] (rright);
    \end{forest}
\end{figure}
\noindent \textbf{Normal form:}
\begin{table}[h!]
    \centering
    \begin{tabular}{cccc}
                                                   &                                 & \multicolumn{2}{c}{Ringo}                                         \\ \cline{3-4} 
                                                   & \multicolumn{1}{c|}{}           & \multicolumn{1}{c|}{Gold Coast} & \multicolumn{1}{c|}{Queenstown} \\ \cline{2-4} 
        \multicolumn{1}{c|}{\multirow{2}{*}{Paul}} & \multicolumn{1}{c|}{Gold Coast} & \multicolumn{1}{c|}{\underline{\textbf{5}}, \underline{\textbf{3}}}       & \multicolumn{1}{c|}{-2, -2}     \\ \cline{2-4} 
        \multicolumn{1}{c|}{}                      & \multicolumn{1}{c|}{Queenstown} & \multicolumn{1}{c|}{-2, -2}     & \multicolumn{1}{c|}{\underline{\textbf{1}}, \underline{\textbf{6}}}       \\ \cline{2-4} 
    \end{tabular}
\end{table}

\subsection*{3.2} Using best response analysis on the Normal Form game table shows that two Nash equilibria exist, these are when either both players choose `Gold Coast', or both players choose `Queenstown'. These equilibria are illustrated on the extensive form game tree in {\color{red}red} and {\color{blue}blue} respectively, note that Ringo is forced to make the same choice at each of his decision nodes since it is a simultaneous game.

\subsection*{3.3} This game has a payoff maximising focal point of (Gold Coast, Gold Coast) since the total payoff for that NE is $5+3=8$ whereas the other NE has total payoff $1+6=7$. This relies on the fact that Ringo is ok with sacrificing personal benefit in order to maximise the total payoff for both players. This seems like a reasonable assumption considering they appear to have a friendly relationship based on the increased payoffs they get from holidaying together.

\subsection*{3.4} The extensive form of this game is as follows:
\begin{figure}[H]
    \centering
    \begin{forest}
        for tree={circle,s sep = 35pt,calign=center}
        [Paul,
         [Ringo,edge={->, red, line width=1.5pt},edge label={node[above left,midway,font=\scriptsize]{Gold Coast}}
          [{(5, 3)},edge={->, red, line width=1.5pt},edge label={node[midway,above left,font=\scriptsize]{Gold Coast}}]
          [{(-2, -2)},edge label={node[midway,above right,font=\scriptsize]{Queenstown}}]
         ]
         [Ringo, edge label={node[midway,above right,font=\scriptsize]{Queenstown}}
          [{(-2, -2)},edge label={node[midway,above left,font=\scriptsize]{Gold Coast}}]
          [{(1, 6)},edge={->, red, line width=1.5pt},edge label={node[midway,above right,font=\scriptsize]{Queenstown}}]
         ]
        ]
    \end{forest}
\end{figure}
\noindent The path of the SPNE is Gold Coast $\rightarrow$ Gold Coast, Paul's SPNE strategy is (Gold Coast), and Ringo's is (Gold Coast, Queenstown). This is illustrated on the game tree in {\color{red}red}.\\[2mm]
The game has a first mover advantage as Paul is able to make Ringo choose Gold Coast (Paul's maximum payoff) since it is detrimental to Ringo to disagree with Paul's decision.

\subsection*{3.5} \paragraph{Note} To save space in this question Gold Coast and Queenstown are abbreviated GC and QT respectively.\\[2mm]
The extensive form game is the following, where each players equilibrium strategies are highlighted in {\color{red}red}.
\begin{figure}[H]
    \centering
    \begin{forest}
        for tree={circle,s sep = 20pt,calign=center}
        [Ringo,
         [Paul, edge={->, red, line width=1.5pt}, edge label={node[above left,midway,font=\scriptsize]{Purchase \textbf{Extreme} Sports Pack}}
          [Ringo,edge label={node[above left,midway,font=\scriptsize]{GC}}
           [{(-2, 5)},edge label={node[midway,above left,font=\scriptsize]{GC}}]
           [{(8, -2)},edge={->, red, line width=1.5pt}, edge label={node[midway,above right,font=\scriptsize]{QT}}]
          ]
          [Ringo, edge={->, red, line width=1.5pt}, edge label={node[midway,above right,font=\scriptsize]{QT}}
           [{(-7, -2)},edge label={node[midway,above left,font=\scriptsize]{GC}}]
           [{(10, 1)},edge={->, red, line width=1.5pt}, edge label={node[midway,above right,font=\scriptsize]{QT}}]
          ]
         ]
         [Paul, edge label={node[above right,midway,font=\scriptsize]{Don't Purchase \textbf{Extreme} Sports Pack}}
          [Ringo,edge={->, red, line width=1.5pt},edge label={node[above left,midway,font=\scriptsize]{GC}}
           [{(3, 5)},edge={->, red, line width=1.5pt}, edge label={node[midway,above left,font=\scriptsize]{GC}}]
           [{(-2, -2)},edge label={node[midway,above right,font=\scriptsize]{QT}}]
          ]
          [Ringo, edge label={node[midway,above right,font=\scriptsize]{QT}}
           [{(-2, -2)},edge label={node[midway,above left,font=\scriptsize]{GC}}]
           [{(6, 1)},edge={->, red, line width=1.5pt}, edge label={node[midway,above right,font=\scriptsize]{QT}}]
          ]
         ]
        ]
    \end{forest}
\end{figure}
\noindent In this situation, the SPNE path is `Purchase \textbf{Extreme} Sports Pack' $\rightarrow$ Queenstown $\rightarrow$ Queenstown. The players equilibrium strategies are:
\begin{itemize}
    \item Ringo: \{(Purchase \textbf{Extreme} Sports Pack, QT, QT, GC, QT)\}; and,
    \item Paul: \{(QT, GC)\}.
\end{itemize}

\subsection*{3.6}
The sports pack \textit{can} be seen as a commitment device, this works because when it comes to Paul's decision, if Ringo has the sports pack, Paul knows that Ringo will be happy to go to Queenstown by himself. This makes Paul pick Queenstown too so that they can enjoy the payoff from both going to the same location.
\end{document}